%pæn (projekttitel på forside!!!)
%vi skal også beskrive interne regler (fx bødekasse)
%sekvensdiagram
%designmockups


\documentclass[11pt,a4paper]{report}

\setlength{\textwidth}{165mm}
\setlength{\textheight}{240mm}
\setlength{\parindent}{0mm} % S{\aa} meget rykkes ind efter afsnit
\setlength{\parskip}{\baselineskip}
\setlength{\headheight}{0mm}
\setlength{\headsep}{0mm}
\setlength{\hoffset}{-2.5mm}
\setlength{\voffset}{0mm}
\setlength{\footskip}{15mm}
\setlength{\oddsidemargin}{0mm}
\setlength{\topmargin}{0mm}
\setlength{\evensidemargin}{0mm}



\usepackage{courier} % The font we use.
\usepackage[a4paper, hmargin={2.8cm, 2.8cm}, vmargin={2.5cm, 2.5cm}]{geometry}
\usepackage{eso-pic} % \AddToShipoutPicture
\usepackage{graphicx} % \includegraphics
\usepackage[english]{babel}
\usepackage[utf8]{inputenc}
\usepackage{appendix}
\usepackage{amsfonts,amsmath,amssymb}
\usepackage[colorinlistoftodos]{todonotes}
%\usepackage{gauss} % Gauss Matrix
\usepackage{float} % This will allow precise picture placement, use [H].
\usepackage{enumitem}

\usepackage{microtype}
\usepackage[super]{nth}
\usepackage{booktabs} % This package provide some additional commands to enhance the quality of tables in LaTeX.
\usepackage{listings} % code parsing.
\PassOptionsToPackage{hyphens}{url}\usepackage{hyperref}
\lstset{language=bash} % Set bash syntax.
\newcommand{\code}[1]{\texttt{#1}}

\newlist{SubItemList}{itemize}{1}
\setlist[SubItemList]{label={$-$}}

\let\OldItem\item
\newcommand{\SubItemStart}[1]{%
    \let\item\SubItemEnd
    \begin{SubItemList}[resume]%
        \OldItem #1%
}
\newcommand{\SubItemMiddle}[1]{%
    \OldItem #1%
}
\newcommand{\SubItemEnd}[1]{%
    \end{SubItemList}%
    \let\item\OldItem
    \item #1%
}
\newcommand*{\SubItem}[1]{%
    \let\SubItem\SubItemMiddle%
    \SubItemStart{#1}%
}%

\newcommand{\BAR}{%
  \hspace{-\arraycolsep}%
  \strut\vrule % the `\vrule` is as high and deep as a strut
  \hspace{-\arraycolsep}%
}

\author{\Large{Sven Frenzel (\href{mailto:sven@frenzel.dk}{sven@frenzel.dk}) - 130793 - cdn769}\\
\Large{Mads Gram (\href{mailto:mgmadsgram@gmail.com}{mgmadsgram@gmail.com})  - 081293 - wtc324}\\
\Large{Thorkil Værge (\href{mailto:thorkilk@gmail.com}{thorkilk@gmail.com}) - 150287 - wng750} \\ \\
\Large{Instructor: Kasper Passov }}

\title{
\vspace{3cm}
\Large{Second Partial Assignment - ProjDat}
}

\begin{document}

%% Change `ku-farve` to `nat-farve` to use SCIENCE's old colors or
%% `natbio-farve` to use SCIENCE's new colors and logo.
\AddToShipoutPicture*{\put(0,0){\includegraphics*[viewport=0 0 700 600]{include/natbio-farve}}}
\AddToShipoutPicture*{\put(0,602){\includegraphics*[viewport=0 600 700 1600]{include/natbio-farve}}}

%% Change `ku-en` to `nat-en` to use the `Faculty of Science` header
\AddToShipoutPicture*{\put(0,0){\includegraphics*{include/nat-en}}}

\clearpage\maketitle
\thispagestyle{empty}

\newpage
\tableofcontents{}
\thispagestyle{empty}


\newpage

\chapter{Literature Review}\label{ch:Literature_Review}

\section{Brooks, F. P. (1986) No silver bullet}
\subsection{Résumé}
\subsection{Analysis and Perspective}
\subsection{Perspectivation to Work on dikukeys}

\newpage
\section{Christensen et.al. (1998) The M.A.D. Experience: Multiperspective Application Development in evolutionary prototyping.}
\subsection{Résumé}
\subsection{Analysis and Perspective}
\subsection{Perspectivation to Work on dikukeys}


\chapter{Project Report}\label{ch:Project_Report}

\section{Abstract}\label{sec:Abstract}
The system developed [system], as part of this project, is a module of a system developed by Oleksander Shturmov [Online TA]. The purpose of the system is to validate the identity of a student submitting an assignment to Online TA. \\
This is done by mapping OpenPGP keys provided by the students with their KU ID. This mapping is achieved through a website on which a student registers themselves. The website is part of the system developed. Furthermore an API is provided to Online TA for integration. The system is developed for the Department of Computer Science at the University of Copenhagen [DIKU] represented by Oleksander Shturmov. Oleksander Shturmov has shown a desire to replace the currently used Learning Management System [LMS] \textit{its Learning} with a platform that can facilitate automatically evaluation of programming assignments. For this he requires a system to make sure that a given assignment has been handed in by the student claiming to have handed it in. This is the system we are developing.
A prototype of the system is available at \url{http://dikukeys.dk}.

\section{Purpose and Framework for the Project}\label{sec:Purpose_Framework}
This definition is based on the FACTOR criteria which serves the purpose of providing the structure for a systems definition.
\begin{itemize}
\item \textbf{F}unctionality: The purpose of the system is to allow specific students and employees\footnote{Everyone with valid KU ID in the form of xyz123 - 'svensk nummberplade'.} at the University of Copenhagen to map their username to a public PGP key and store these records in a database maintained by the customer. The system must also allow the same users to replace their public key in case they lose the corresponding private key. The mappings should be readable by specific users with privileged access through some login mechanism.
There should also exist an automatic way for other systems to access the mappings in a structured manner.
\item \textbf{A}pplication domain: The system will be used by students to upload their OpenPGP key and by teaching assistants to match hand-ins that have been signed by a student's OpenPGP key with a KU ID. It is used when TAs receive hand-ins by the students and before the grading commences.
\item \textbf{C}onditions: Most students are not likely to have experience with OpenPGP so it should be as simple as possible to upload a key and the system must be forgiving, meaning that if a student loses their private PGP key, they must be able to upload a new public key. Instructions to guide the students through the various tasks of creating and uploading a key must be available. The system must allow other software to access the mappings.
\item \textbf{T}echnology: Servers with a UNIX-like OS will run the system, and modern browsers will be used to display the front-end. The system will be written in golang to the widest extent possible but will also utilize SQL, HTML, CSS, and JS. It will also use existing software in the form of nginx for the http-server and existing golang libraries for example sqlite3.
\item \textbf{O}bjects: The primary objects in the problem domain are students, TAs, their KU ID, and public OpenPGP keys.
\item \textbf{R}esponsibility: The system's main responsiblity is to maintain a table with a mapping from KU IDs to corresponding OpenPGP public keys. It must also allow other privileged software and privileged users to read from this table.
\end{itemize}

\renewcommand{\thesubsection}{\thesection.\alph{subsection}}
\section{Specification of Requirements for the IT Solution}\label{sec:Requirements}
\subsection{Functional and Non-functional Requirements}
\subsubsection{Functional Requirements}\label{subsubsec:Functional_Req}
The functional requirements describe features which are essential to the success of the project and describe actual functionality for the end-user or system administrator.
\begin{itemize}
\item The system must map a KU ID to an OpenPGP public key.
\item Users must be able to upload their own public key to the service.
\item If a user loses their private key, a method for replacing the public key must exist.
\item The initial registration of a key, and the replacement of a key, is authenticated through the KU email system.
\item The data has to be accessible to privileged software through a defined API.
\item The mappings have to be available to privileged users through some means.
\item The KU IDs registered in this system must \textbf{not} be accessible for non-privileged users and software.
\item A guide for creating key pairs needs to be available on the website, possibly as shell script. This guide should also explain how to use an OpenPGP key for SSH authentication.
\item A way for an administrator to send email invitations for joining the system to a list of students must exist.
\end{itemize}

\subsubsection{Non-functional Requirements}\label{subsubsec:Non_Functional_Req}
Non-functional requirements are system requirements which do not describe the actual functionality but instead describe either non-essential systems, internal architecture, or design.
\begin{itemize}
\item The back-end should be written in golang.
\item The front-end should be Javascript and HTML5.
\item The system must be browser independent\footnote{This includes recent versions of Internet Explorer, Chrome, Safari and Firefox.}.
\item The system must be compliant with all standards and regulations imposed by the Government of Denmark and the University of Copenhagen. These regulations concern privacy and restriction of access to the KU IDs that are part of the system. Specifically, the KU IDs in the system must not be accessible to non-privileged users\footnote{The priveliged users are system administrators, university professors, and TAs.}.
\item The customer has decreed that the software be licensed under an MIT-like license which will be provided by the customer\footnote{A preliminary version of this license is available at \href{https://github.com/oleks/sandstone/blob/master/LICENSE}{https://github.com/oleks/sandstone/blob/master/LICENSE}.}.
\item The system should be able to decode the different .csv-formats (comma, semicolon, tab) for student lists.
\item The front-end must be responsive in its layout. This is achieved by using Bootstrap\footnote{Bootstrap can be found on \href{http://getbootstrap.com/}{http://getbootstrap.com/}}.
\end{itemize}

\subsection{Use Case Overview Model}\label{subsec:Use_case_model}

Figure \ref{fig:use_case_diagram_high_level} represents a high-level model which shows which actions are available for the different users: students and teachers.
A student can upload and replace their public key and the TA can read the mappings.

\begin{figure}[H]
\centering
\includegraphics[width=0.5\textwidth]{pictures/use_case_pksu_del2_b_high}
\caption{High-level use case diagram. Students can upload or replace a public key, and teachers (TAs, professors, and system administrators) can pull data from the lists.}
\label{fig:use_case_diagram_high_level}
\end{figure}


\subsection{Specific Use Case Models}\label{subsec:Specific_Use_case_model}
Here, three different use cases for the system is shown by listing the actions of the relevant actors. The use cases for uploading a key, replacing a key, and sending out an invitation for joining the system are listed.
\subsubsection{Use Case: Upload Key}
\begin{tabular}{l p{0.8\textwidth}}
    \toprule
    \textit{Use case name} & Upload Key \\
    \midrule
    \textit{Participating} & Students \\
    \textit{actors} & \\
    \midrule
    \textit{Flow of events} &
    \vspace{-6.7mm} \begin{enumerate}
        \item Student follows link to dikukeys.dk in invitation email OR student goes to dikukeys.dk directly.
        \item Student enters their KU ID into a form.
        \item Server sends activation email to student's KU email address.
        \item Student follows link in activation email.
        \item Student posts their public key into a form or generates a key pair in the browser.
        \item Server sends email which confirms that a key has been uploaded.
    \end{enumerate}
    \\
    \midrule
    \textit{Entry condition} & User has an active KU ID. \\
                             & User is not registered in the system. \\
    \midrule
    \textit{Exit conditions} & User is registered in the system with a public key. \\
    \bottomrule
\end{tabular}

\subsubsection{Use Case: Replace Key}
\begin{tabular}{l p{0.8\textwidth}}
    \toprule
    \textit{Use case name} & Replace Key \\
    \midrule
    \textit{Participating} & Students \\
    \textit{actors} & \\
    \midrule
    \textit{Flow of events} &
    \vspace{-6.7mm} \begin{enumerate}
        \item Student goes to dikukeys.dk.
        \item Student enters their KU ID into a form.
        \item Server sends key replacement email to student's KU email address.
        \item Student follows link in key replacement email.
        \item Student posts their new public key into a form or generates a new key pair in the browser.
        \item Server sends email which confirms that the new key has been uploaded.
    \end{enumerate}
    \\
    \midrule
    \textit{Entry condition} & Student is already a registered user in the system without the private key corresponding to the listed public key. \\
    \midrule
    \textit{Exit conditions} & User is registered in the system with a new public key. \\
    \bottomrule
\end{tabular}

\subsubsection{Use Case: Invite Group of Students}
\begin{tabular}{l p{0.8\textwidth}}
    \toprule
    \textit{Use case name} & Invite Group of Students\\
    \midrule
    \textit{Participating} & Course director \\
    \textit{actors} & System administrator \\
    \midrule
    \textit{Flow of events} &
    \vspace{-6.7mm} \begin{enumerate}
        \item Course director sends a list of students and information about course to system administrator.
        \item System administator submits the list of students and the course information to the server via ssh.
        \item Server checks if students already are in the system. Depending on the answer one of the following two happens:
        \item
        \begin{enumerate}
            \item If the student is \textbf{not} already registered, the server sends an invitation email to the KU email address of the student.
            \item If the student is already registered, the server sends an email to the student informing them that the system will be used in this course.
        \end{enumerate}
    \end{enumerate}
    \\
    \midrule
    \textit{Entry condition} & User is course director. \\
                             & User has a list of students. \\
                             & \nth{2} user is system administrator. \\
    \midrule
    \textit{Exit conditions} & All students from the list have either received an invitation to the system or have been informed that the system will be used in this course. \\
    \bottomrule
\end{tabular}

\subsection{Class Diagram of Solution Domain}\label{subsec:class_diagram}
The class diagram describes the interacting classes in the solution domain of the system. The classes of the solution domain have been identified as grading by a TA, course, assignment, student, OpenPGP keys, and KU IDs. These categories describe the classes that the users encounter and do thus not describe the system itself. 
\begin{figure}[H]
    \centering
    \includegraphics[width=\textwidth]{pictures/class_diagram_genaflevering}
    \caption{Class diagram of solution domain. The DIKU Keys system interacts with grading, courses, assignments, students, public keys, and KU IDs. A few other classes such as course director could also have been added but in this drawing course directors should be viewed as part of the TAs or part of the course. Each student is assumed to have exactly one OpenPGP Key and one KU ID.}
    \label{fig:class_diagram}
\end{figure}
\subsection*{BCE Model}
The BCE model is shown in Figure \ref{fig:bce_model}. The BCE model is the \nth{1} attempt to describe the actual system (AKA application) domain since the previous class diagram was of the solution domain. The entitiy classes hold the data stored in the system, the control classes describe the logic of the system, and the boundary classes describe the part of the system that interacts with the end-users: the website and the email system. TA API could also be included here. Two models for the control classes have been described as a final choice has not been made yet.
\begin{figure}[H]
    \centering
    \includegraphics[width=\textwidth]{pictures/bce_genaflevering}
    \caption{The BCE model is the \nth{1} part of the systems design. A final choice for the control classes has not been made yet.}
    \label{fig:bce_model}
\end{figure}
\subsection{Sequence Diagrams of Use Case}\label{subsec:Sequence_diagram_Use_case_model}
Figure \ref{fig:sequence_diagram} shows the logical flow between the logical classes identified in the class diagram above. Only the boundary classes can interact directly with the user.
\begin{figure}[H]
    \centering
    \includegraphics[width=\textwidth]{pictures/sequence_diagram_upload}
    \caption{The sequence diagram for the uploading of a public key to the database. Only the boundary classes interact with the user and only the entity class (the database) can permanently change the state of the system.}
    \label{fig:sequence_diagram}
\end{figure}

In figure \ref{fig:use_case_diagram_example_two} we have modeled the more specific use case for uploading a key.

\begin{figure}[H]
    \centering
    \includegraphics[width=\textwidth]{pictures/sequence_diagram_replace}
     \caption{The sequence diagram for the replacement of a public key on the database.}
    \label{fig:use_case_diagram_example_two}
\end{figure}

\section{Resumé of system design}\label{sec:Systemdesign}
\textbf{Implemented:}
\begin{itemize}
\item nginx serves as http-server and communicates with the application
through FastCGI.
\item The application communicates with postfix over SMTP to serve emails.
\end{itemize}
\textbf{Not implemented yet:}
\begin{itemize}
    \item serving a website with adaptive layout
    \item communicating with a database
    \item providing an API with authentication
\end{itemize}


\section{Program- and system tests}\label{sec:Program_systemtests}

\section{User interface and interaction design}\label{sec:UI_interactiondesign}
\begin{enumerate}[label=(\alph*)]
\item Screenshots
The Figure \ref{fig:ss_kuid} shows the initial view a user is met with. It has a fairly simple interface with a form for inputting the user's KU ID and a submit button which should be clicked on to get to the next view.
\begin{figure}[H]
\centering
\includegraphics[width=\textwidth]{pictures/screenshots/kuid}
\caption{Landing page on dikukeys.dk. This is where the user enters their KU ID.}
\label{fig:ss_kuid}
\end{figure}
\begin{figure}[H]
\centering
\includegraphics[width=\textwidth]{pictures/screenshots/email}
\caption{The webmail interface of KU. The registration email has been received by the user.}
\label{fig:ss_email}
\end{figure}
\begin{figure}[H]
\centering
\includegraphics[width=\textwidth]{pictures/screenshots/publickey}
\caption{Page where the user submits their public key. This link is sent in the email above.}
\label{fig:ss_pubkey}
\end{figure}
\item Flow / Dynamics
\item YouTitty video
\item Think out loud user test
\end{enumerate}

\section{Group Dynamics}
\subsection{What is Good?}
The knowledge level of the group seems high considering that we are \nth{1} year students. The group members complement each other well since Sven Frenzel is a smart and fast learner with an eye for detail, especially when it comes to the logic of systems. Frenzel has undertaken the vast majority of the server configuration. Mads Gram has a lot of knowledge in web-design and developing websites, in addition he has made many of the technical diagrams. Thorkil Værge has a good overview and often delegates work, while still doing a significant amount of the textual work, since his written English is remarkable.

Since the talents of the group are dissimilar, when the group works together, we are able to approach the subject from many different perspectives.

The company and social dynamics in the group are pleasant. We work very well when we are brainstorming and dividing up the work amongst ourselves.

\subsection{What is Bad?}
The work process on the \nth{2} report has not been as good as the process for the \nth{1} report since the group has not met and worked together more than one whole day. The rest of the work has been done individually or in groups of two. A better report would have been made if we had dedicated one more day to this report.

Thorkil Værge feels that we use too much time on reports and not enough time on programming and actual systems design. We do, however, learn something from all the different models for the system that we have to make.

\subsection{How can this be improved?}
To improve our work process, we should have more days where we do nothing but work on this project. Experience shows us that these days are the most effective way to work.

\newpage
\begin{appendices}
\chapter{Version Control}

We have chosen to use Git as our version control solution, this we have pared with the only git repository host GitHub.

As earlier agreed upon with our instructor Kasper Passov, we have added him to our Git repository. This will allow him to access our source code.

\href{https://github.com/Orkeren/DIKU-Keyserver/commit/5e7c7c741fbe6d2e74889e8eaff6173e8c116d46}{Inital commit on github.}

This is the initial commit with focus on our source code. It is divided into three parts, app, hash, and mail. App.go is the overarching module, which controls the web-application, the function of App.go is to interface with the web-server (nginx) and the other modules.

hash.go is a go module which enables creation of cryptographic hashes, in this iteration of the code, we only printed it to the console.  The file mail.go is another module which the code uses, this is used to send out mails, in this iteration we used it to send out a mail to our dikumail group, to show how far we were.

\href{https://github.com/Orkeren/DIKU-Keyserver/commit/690f4b0b00dc68e9b529035d7e6e6a6005072d63}{Using HTML templates commit on github.}

Earlier we used pre-encoded HTML in our app.go, with modification and extensive use of the html/template library. We were able to use html templates rather. This allows us to make changes to the pages which are displayed, without stopping the service.

\chapter{Changelog for report}

Since we use Git for our entire project, a timeline of a specific file is possible, so for our report the timeline consists of a link to a history of our document, which our document has been built from. Since we all have individual GitHub accounts, it is possible to see who committed what.

\href{https://github.com/Orkeren/DIKU-Keyserver/commits/master/docs/delrapport_2/aflevering2.tex}{Aflevering2.tex on github.}


\chapter{Timeline}
\begin{itemize}
  \item  March 16th 2015: First meeting with client, initial establishment of project. We agreed upon programming language and choice for authentication Open-PGP keys.
  \item  March 22th 2015: Second meeting with client, here we went into specifics alike API for staff and license
  \item March 29th 2015: Third meeting with client, we agreed upon using nginx as our webserver and some discussion about which database was started, but a final agreement was not reached. Lastly we agreed upon using the Bootstrap framework for the front end design. We setup a test server the same day, and got our first prototype ready.
\end{itemize}

\chapter{Minutes}
\section{Minutes from March 16th 2015}
\begin{itemize}
\item Ønsker: identificer studerende ved aflevering vha gpg-nøgle
\item Ønsker: grube med gpg-nøgle og ku-mail knyttet sammen (da der stoles på at den studerende har adgang til sin email)
\item Autentificering sker igennem unikke links til ku-mail (for at lægge gpg-nøgle op).
\SubItem 2 muligheder: copy/paste gpg-nøgle eller generer gpg-nøglepar direkte i browseren.
\item DIKU leverer ssl-certifikat.
\item Hvorfor ikke bruge eksisterende?
\item Public index j/n?
\item modtager ukrypteret nøgle og bruger den til at verificere indholdet - bagefter checkes hashet pubkey mod databasen.
\item Arbejdsbelastning: 10 timer per person per uger
\item database: (\underline{ku-nummerplade},hash(pubkey))
\end{itemize}
\subsubsection{Krav:}
\begin{itemize}
\item registrering gennem webformular og/eller kommandelinje
\item skal kunne genregistrering
\item anmodning kodkendes igennem KU email
\item kunne vælge om den offentlige nøgle offentliggøres
\SubItem private by default
\item private API: få offentlig nøgle og returner brugernavn (kun adgang til DIKU-ansatte)
\item Sprog
\SubItem Backend: golang + sql
\SubItem frontend: html5 + js
\end{itemize}

næste møde:
2015-03-22 kl 11-16

\section{Minutes from March 22nd 2015}
The meeting was held to align the expectations of the client and the developers in regards to the systems capabilities and design.

\begin{itemize}
\item Oleks undersøger interface hvilke tilgangsmuligheder, der ønskes til databasen.
\item Kun lukket API, intet offentligt look-up. Der gemmes pupkey og abc123 (ku-brugernavn).
\item key-server i stil med MIT-keyserver er nice-to-have. Man bliver kun listet her, hvis man eksplicit ønsker dette.
\item Den studerende uploader en fil til afleveringssystemet (uden anden information). Afleveringssystemet anmoder om par af PubKey \& KU-ID og undersøger om filen er signeret af denne nøgle. Hvis ikke, så anmodes næste par indtil det korrekte par er fundet. Dette returneres til underviser.
\item Beta klar til start/midt maj.
\item Oleks gives a license (MIT-like)
\item ISA-diagram
\SubItem Fjern Admin-login fra (der er sshd-adgang til serveren anyways.)
\SubItem Front-end til client, er response og ikke request
\SubItem User guide /shell script ved ``Create key''
\item Oleks anmoder om at der sendes en dagsorden 24 timer før fremtidige møder
\item Thorkil foreslår at møder optages auditivt.
\end{itemize}

Næste møde er 29/30 marts 2015.
\end{appendices}


\begin{thebibliography}{1}

\bibitem{matthiassen} L. Mathiassen, A. Munk-Madsen, P.A. Nielsen, and J. Stage. {\em Object-Oriented Analysis \& Design} 2000: Marko Publishing House.

\bibitem{beck} Kent Beck et al. 2001 Agile Manifesto. [ONLINE] Available at: http://agilemanifesto.org/principles.html. [Accessed 19 April 15].

\bibitem{gould} Gould \& Lewis. (1985), Designing for usability: key principles and what designers think. Commun. ACM 28, 3 (March 1985), 300-311.

%\bibitem{impj}  The Japan Reader {\em Imperial Japan 1800-1945} 1973:
%Random House, N.Y.

%\bibitem{norman} E. H. Norman {\em Japan's emergence as a modern
%state} 1940: International Secretariat, Institute of Pacific
%Relations.

%\bibitem{fo} Bob Tadashi Wakabayashi {\em Anti-Foreignism and Western
%Learning in Early-Modern Japan} 1986: Harvard University Press.

\end{thebibliography}



\end{document}
