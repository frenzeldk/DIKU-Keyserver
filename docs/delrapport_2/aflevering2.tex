%pæn (projekttitel på forside!!!)
%vi skal også beskrive interne regler (fx bødekasse)
%sekvensdiagram
%designmockups


\documentclass[11pt,a4paper]{report}

\setlength{\textwidth}{165mm}
\setlength{\textheight}{240mm}
\setlength{\parindent}{0mm} % S{\aa} meget rykkes ind efter afsnit
\setlength{\parskip}{\baselineskip}
\setlength{\headheight}{0mm}
\setlength{\headsep}{0mm}
\setlength{\hoffset}{-2.5mm}
\setlength{\voffset}{0mm}
\setlength{\footskip}{15mm}
\setlength{\oddsidemargin}{0mm}
\setlength{\topmargin}{0mm}
\setlength{\evensidemargin}{0mm}



\usepackage{courier} % The font we use.
\usepackage[a4paper, hmargin={2.8cm, 2.8cm}, vmargin={2.5cm, 2.5cm}]{geometry}
\usepackage{eso-pic} % \AddToShipoutPicture
\usepackage{graphicx} % \includegraphics
\usepackage[english]{babel}
\usepackage[utf8]{inputenc}
\usepackage{amsfonts,amsmath,amssymb}
\usepackage[colorinlistoftodos]{todonotes}
%\usepackage{gauss} % Gauss Matrix
\usepackage{float} % This will allow precise picture placement, use [H].
\usepackage{enumitem}
\usepackage{hyperref} % Url's
\usepackage{microtype}
\usepackage{booktabs} % This package provide some additional commands to enhance the quality of tables in LaTeX.
\usepackage{listings} % code parsing.
\lstset{language=bash} % Set bash syntax.
\newcommand{\code}[1]{\texttt{#1}}

\newlist{SubItemList}{itemize}{1}
\setlist[SubItemList]{label={$-$}}

\let\OldItem\item
\newcommand{\SubItemStart}[1]{%
    \let\item\SubItemEnd
    \begin{SubItemList}[resume]%
        \OldItem #1%
}
\newcommand{\SubItemMiddle}[1]{%
    \OldItem #1%
}
\newcommand{\SubItemEnd}[1]{%
    \end{SubItemList}%
    \let\item\OldItem
    \item #1%
}
\newcommand*{\SubItem}[1]{%
    \let\SubItem\SubItemMiddle%
    \SubItemStart{#1}%
}%

\newcommand{\BAR}{%
  \hspace{-\arraycolsep}%
  \strut\vrule % the `\vrule` is as high and deep as a strut
  \hspace{-\arraycolsep}%
}

\author{\Large{Sven Frenzel (\href{mailto:sven@frenzel.dk}{sven@frenzel.dk}) - 130793 - cdn769}\\
\Large{Mads Gram (\href{mailto:mgmadsgram@gmail.com}{mgmadsgram@gmail.com})  - 081293 - wtc324}\\
\Large{Thorkil Værge (\href{mailto:thorkilk@gmail.com}{thorkilk@gmail.com}) - 150287 - wng750} \\ \\
\Large{Instructor: Kasper Passov }}

\title{
\vspace{3cm}
\Large{Second Partial Assignment - ProjDat}
}

\begin{document}

%% Change `ku-farve` to `nat-farve` to use SCIENCE's old colors or
%% `natbio-farve` to use SCIENCE's new colors and logo.
\AddToShipoutPicture*{\put(0,0){\includegraphics*[viewport=0 0 700 600]{include/natbio-farve}}}
\AddToShipoutPicture*{\put(0,602){\includegraphics*[viewport=0 600 700 1600]{include/natbio-farve}}}

%% Change `ku-en` to `nat-en` to use the `Faculty of Science` header
\AddToShipoutPicture*{\put(0,0){\includegraphics*{include/nat-en}}}

\clearpage\maketitle
\thispagestyle{empty}

\newpage
\tableofcontents{}
\thispagestyle{empty}


\newpage

\chapter{Literature Review}\label{ch:Literature_Review}

\section{Gould \& Lewis -- Designing for Usability: key principles and what
designers think}
\subsection{Resume}
This article from 1985 can be divided into two different parts: Actual recommendations for systems design and a survey to estimate how often this process is followed. The principles for good software design are outlined as follows:
\begin{itemize}
\item Early focus on users: designers must understand who the users will be, and this understanding is preferably obtained by actual interaction between the developer and end user.
\item Empirical measurement: Early in the process intended users should use simulations and prototypes to carry out real work and their performance should influence the system through
\item Interactive design: Problems during for example testing should influence design decisions. 
\end{itemize}
The survey finds that most programmers do not mention these three points and when they do, they often misunderstand or misrepresent the actual meaning of the points. They may think that they are following the recommendations but they are not. Several reasons that the recommendations are not followed is listed among these are that the programmers sometimes do not take responsibility for the entire project but only for the modules that are assigned to them. The article also emphasises the limitations of reason and stresses the need for empirical data in design decisions. This is illustrated by some good but counterintutive design decisions from the design of a voice mail system from IBM called ADS.

Behavioral goals should be developed for the end user and these goals should be testable. A goal could for example be that 80 \% of the users should be able to solve some task within a given amount of time without help or other instructions than what is given through the interaction with the system. Modularization is also recommended.
\subsection{Analysis and Perspective}
This article differs from other litterature that we have read in that it focuses on the limitations of reason and the need for empirical observations when making design decisions. It stresses that the optimal design choice can in some situations not be deducted but must be found through interaction with end users. On page 306 the article makes a very bold prediction that ease of use will become important when selling computer devices. The fulfillment of that prophecy has, among others, been achived by Apple with their iPod and iPhone.
\subsection{Perspectivation to work on dikukeys}
This article focuses a lot on interaction with the end user. Actual user testing has not been done during the work on dikukeys and it is probably not possible to do before a minimally working website is done. It is something that was not even on our list of planned tasks but that is something we should definitely include in the system. A user test can still influence the design process even though it is performed late since we follow an iterative process (AGILE). 

\section{Parnas \& Clements -- A Rational Design Process: How and why to
fake it}
\subsection{Resume}
The article explains that a perfectly rational design process would be preferable so that a computer program or service could be derived rationally in the same sense that theorems are derived from axioms in mathematics. But for several reasons, this process cannot be followed perfectly: customers do not know their exact requirements before some of the functionality is presented, during the programming new discoveries are made which invalidate previous beliefs so we must backtrack, the complexity is too overwhelming to include all elements before programming and testing is initiated, and unpredictable external changes may occur.

Nonetheless, there are still many reasons for attempting to follow a rational design process. Some of the reasons which are mention in the article are: simply attempting makes us come closer to the rational design process and reduces backtracking and standardization of the design process makes it easier for employees to join an ongoing project.

The rational design process is defined by the work products:
\begin{itemize}
\item A requirements document: a place to record the desired behavior of the program.
\item A module structure: A division of the program and work programming into logical subentities. Preferably made as a tree structure.
\item Module interfaces: Formalized description of how the modules interact, an interface specification. Like an API for each module.
\item Uses hierarchy: A binary matrix which lists all the dependencies of the individual modules.
\item Internal module structure: Documentation of design decisions for each module defined in module structure.
\item The actual program: The source code of the program for delivery. Should not contain redundant code described in the documentation mentioned above.
\end{itemize}
If any changes during the design phase, programming phase, or after delivery invalidate a design document, then all the documentation must be ``faked'' to look as if the change had been the original design. This design documentation is similar to a mathematical proof where the logical design and not the temporal development is the guiding force. Repetitions should be avoided in the documentation by asking: ``what questions should this specific document answer'', before it is written.
\subsection{Analysis and Perspective}
If this design process is followed then the documentation will follow the logical structure outlined above. Decisions made whilst coding or testing can feed into changes in the requirements document so the design process becomes iterative. This document thus encompasses many important elements of the Agile Manifesto\cite{beck} written 15 years later. For some of the group members, a temporal process in the documentation instead of a logical guiding force seems more intuitive but this article makes a very convincing case why the logical structure is preferable both during the programming phase and especially during the maintenance phase.
\subsection{Perspectivation to work on dikukeys}
Due to the very rigid and comprehensive structure of these partial assignments, not all recommendations in this article can be followed: Many of the sections we are asked to write forces us to repeat ourselves which should be avoided according to this article. The repetition for instance occurred in the ``FACTOR analysis'' of partial assignment number 1. To follow the recommendation in this , the repetitions should be avoided in the final report. A positive effect of the many, sometimes redundant, questions we are asked to answer, is that we become familiar with many different concepts and work methods which we would not have been taught if we had followed the recipe outlined in this article. After reading this article, we were motivated to follow this design process rigorously but it is not certain that the requirements set forth by the teachers will allow us to do that in the final report.


\chapter{Project Report}\label{ch:Project_Report}

\section{Abstract}\label{sec:Abstract}

\section{Purpose and Framework for the project}\label{sec:Purpose_Framework}
This definition is based on the FACTOR criteria which serves the purpose of providing the structure for a systems definition.
\begin{itemize}
\item \textbf{F}unctionality: The purpose of the system is to allow specific students and employees at the University of Copenhagen to map their username to a public PGP key and store these records in a database maintained by the customer. The system must also allow the same users to replace their public key in case they lose the corresponding private key. The mappings should be readable by specific users with privileged access through some login mechanism.
\item \textbf{A}pplication domain: The system will be used by students to upload their OpenPGP key and by teaching assistants to match hand-ins that have been signed by a student's OpenPGP key. It is used when TAs receive hand-ins by the students and before the grading commences.
\item \textbf{C}onditions: Most students are not likely to have experience with OpenPGP so it should be as simple as possible to upload a key and the system must be forgiving, meaning that if a student loses his private PGP key, they must be able to upload a new public key. Instructions to guide the students through the various tasks of creating and uploadning a key must be available. The system must allow other software to access the mappings.
\item \textbf{T}echnology: Servers with a UNIX-like OS will run the system, and modern browsers will be used for the front-end. The system will be written in golang to the widest extent possible but will also utilize SQL, HTML, CSS as different languages, and also use existing software in the form of nginx for the webserver.
\item \textbf{O}bjects: The primary objects in the problem domain are students, TAs, and their uniquely identifying keys, in this case OpenPGP-keys.
\item \textbf{R}esponsibility: The system's main responsiblity is to maintain a table with a mapping from KU Usernames to corresponding OpenPGP public keys. It must also allow other privileged software to read from this table.
\end{itemize}

\renewcommand{\thesubsection}{\thesection.\alph{subsection}}
\section{Specification of Requirements for the IT Solution}\label{sec:Requirements}
\subsection{Functional and Non-Functional Requirements}
\subsubsection{Functional Requirements}\label{subsubsec:Functional_Req}
The functional requirements describe features which are essential to the success of the project and describe actual functionality for the end-user or system administrator. The following are deemed functional requirements:
\begin{itemize}
\item The system must map a KU Username to an OpenPGP public key.
\item Users must be able to upload their own public key to the service.
\item If a user loses their private key, a method for replacing the public key must exist.
\item The initial registration of a key, and the replacement of a key, is authenticated through the KU email system.
\item The data has to be accessible to privileged software through a defined API.
\item The KU usernames registered in this system must not be accessible for non-privileged users and software.
\item A guide for creating key pairs needs to be available on the website, possibly as shell script. This guide should also explain how to derive an SSH key from an OpenPGP key.
\item A way for an administrator to send email invitations to join the system to a list of students.
\end{itemize}

\subsubsection{Non-functional Requirements}\label{subsubsec:Non_Functional_Req}
Non-functional requirements are system requirements which do not describe the actual functionality but instead describe either non-essential systems, internal architecture, or design.
\begin{itemize}
\item The back-end should be written in golang.
\item The front-end should be Javascript and/or HTML5, which ever is safest and most easy to use.
\item The system should be browser independent.
\item The system must be compliant with all standards and regulations imposed by the Government of Denmark and the University of Copenhagen. These regulations concern privacy and restriction of access to the KU usernames that are part of the system. Specifically the KU usernames in the system must not be accessible to non-privileged users\footnote{This will be relevant system administrators, lectors, and TAs.}.
\item The customer has decreed that the software be licensed under a MIT-like license which will be provided by the customer.
\item The system should be able to decode the different .csv-formats (comma, semicolon, tab) for student lists.
\item The front-end must be responsive in its layout. This is achieved by using bootstrap\footnote{\href{http://getbootstrap.com/}{http://getbootstrap.com/}}.
\end{itemize}

\subsection{Use case model}\label{subsec:Use_case_model}

Figure \ref{fig:use_case_diagram_high_level} represents a high-level model, with a emphasis on defining actions for actors.

% Think about moving text and diagram under the specific use case, rather than here!
In figure \ref{fig:use_case_diagram_example} we have modeled the more specific use case for uploading a key.

\begin{figure}[H]
\centering
\includegraphics[width=0.7\textwidth]{pictures/use_case_pksu_del2_b_high}
\caption{High-level use case diagram.}
\label{fig:use_case_diagram_high_level}
\end{figure}


\subsection{Specific use case models}\label{subsec:Specific_Use_case_model}

\subsubsection{Use case: Upload key}
\begin{tabular}{l p{0.8\textwidth}}
    \toprule
    \textit{Use case name} & Upload key \\
    \midrule
    \textit{Participating} & Students \\
    \textit{actors} & \\
    \midrule
    \textit{Flow of events} &
    \vspace{-6.7mm} \begin{enumerate}
        \item Student follows link to dikukeys in invitation email OR student goes to dikukeys directly.
        \item Student enter their KU Username into a form.
        \item Server sends activation email to student's KU-email address.
        \item Student follows link in activation email.
        \item Student posts their public key into a form or generates a key pair in the browser
        \item Server sends email which confirms that a key has been uploaded.
    \end{enumerate}
    \\
    \midrule
    \textit{Entry condition} & User has an active KU-ID. \\
                             & User is not registered in the system. \\
    \midrule
    \textit{Exit conditions} & User is registered in the system with a public key. \\
    \bottomrule
\end{tabular}

-In figure \ref{fig:use_case_diagram_example} we have modeled the more specific use case for uploading a key.
-
\begin{figure}[H]
    \centering
    \includegraphics[width=0.7\textwidth]{pictures/use_case_pksu_del2_b_example}
    \caption{Specific example of use case for upload key.}
    \label{fig:use_case_diagram_example}
\end{figure}

\subsubsection{Use case: Replace key}
\begin{tabular}{l p{0.8\textwidth}}
    \toprule
    \textit{Use case name} & Replace key \\
    \midrule
    \textit{Participating} & Students \\
    \textit{actors} & \\
    \midrule
    \textit{Flow of events} &
    \vspace{-6.7mm} \begin{enumerate}
        \item Student goes to dikukeys.
        \item Student enter their KU Username into a form.
        \item Server sends replacement email to student's KU-email address.
        \item Student follows link in replacement email.
        \item Student posts their new public key into a form or generates a new key pair in the browser
        \item Server sends email which confirms that the new key has been uploaded.
    \end{enumerate}
    \\
    \midrule
    \textit{Entry condition} & Student is already registered user in the system. \\
    \midrule
    \textit{Exit conditions} & The new key has been registered in the system. \\
    \bottomrule
\end{tabular}

\subsubsection{Use case: Invite group of students}
\begin{tabular}{l p{0.8\textwidth}}
    \toprule
    \textit{Use case name} & Invite group of students\\
    \midrule
    \textit{Participating} & Course director \\
    \textit{actors} & System administrator \\
    \midrule
    \textit{Flow of events} &
    \vspace{-6.7mm} \begin{enumerate}
        \item Course director sends list of students and information about course to system administrator.
        \item System administator submits list of students course-information to the server via ssh.
        \item Server checks if students already are in the system. Depending on the answer one of the following two happens:
        \item
        \begin{enumerate}
            \item If the student is \textbf{not} already registered, the server sends invitation email to the KU-email address of the student.
            \item If the student is already registered, the server sends an email to the student informing them that the system will be used in the course.
        \end{enumerate}
    \end{enumerate}
    \\
    \midrule
    \textit{Entry condition} & User is course director \\
                             & User has a list of students \\
    \midrule
    \textit{Exit conditions} & All students from the list have either received an invitation to the system or have been informed that the system will be used in the course. \\
    \bottomrule
\end{tabular}

\subsection{Class diagram of solution domain}\label{subsec:class_diagram}
A class diagram, where five different classes has been identified, has been made. The classes which are contained constitute different parts of software as well as the server on which the system is hosted. The classes have been identified according to the BCE model. The BCE analysis is especially relevant for the sequence diagram since it shows which classes can interact with the user. Only the boundary (B) classes can do that. And only interactions with the entity classes can change the state of the system through a user input.
\begin{figure}[H]
    \centering
    \includegraphics[width=1.2\textwidth]{pictures/class_diagram}
    \caption{Class diagram of solution domain. The dikuserver solution consists of the classes which normally consitutes a web solution: the http handler (web server), the logic which creates the HTML code, the database which can permanently record user input, and a webserver which is an alternative way of communicating with the user and whose purpose it is to validate the identity of the user.}     -    \label{fig:class_diagram}
\end{figure}
\subsection*{}

\subsection{Sequence Diagrams of use case}\label{subsec:Sequence_diagram_Use_case_model}

\begin{figure}[H]
    \centering
    \includegraphics[width=1.2\textwidth]{pictures/sequence_diagram_upload}
    \caption{The sequence diagram for the uploading of a public key to database. Only the boundary classes interact with the user and only the entity class (the database) can permanently change the state  of the system.}
    \label{fig:sequence_diagram}
\end{figure}

In figure \ref{fig:use_case_diagram_example} we have modeled the more specific use case for uploading a key.

\begin{figure}[H]
    \centering
    \includegraphics[width=1.2\textwidth]{pictures/sequence_diagram_replace}
     \caption{Specific example of use case for upload key.}
    \label{fig:use_case_diagram_example}
\end{figure}

\section{Systemdesign sammenfatning}\label{sec:Systemdesign_sammenfatning}

\section{Program- and systemtests}\label{sec:Program_systemtests}

\section{Brugergrænseflade og interaktionsdesign}

\section{Projektsamarbejdet}

\section{Project Definition}
\subsection{Overview of the Customer and the Project}
% vores overview kræver knowledge om nøgler - vi skal skrive ``vores system skal sikre at filerne kommer fra person abc123''
The Ph.D student Oleksandr Shturmov at the Department of Computer Science, University of Copenhagen, has presented us with the project of mapping KU usernames to public OpenPGP keys and creating the guides for the students and writing the server backend and frontend to accomplish this. This system should be used by the students, the teaching assistants, and the teachers at the university to ensure the authenticity of the identity of the end-user, that is, the student.
\subsection{Problem statement}
\subsubsection{Problem Domain}
At the University, many courses require the students to hand in assignments. In this modern era, this is done by electronic hand-ins over the World Wide Web on the interconnected global network. It is essential to ensure that there can be no doubt of the identity of the student handing in the assignment because the teachers grade the students based on these electronic hand-ins.

This problem can be solved by a classic login system where the user has a unique username and picks a password as a secret piece of information to avoid malicious parties to get access to user privileges. This system has already been implemented by the Norwegian company its Learning through the World Wide Web platform ``Absalon''.

Another option, which is the subject of this project, is to use asymmetric key encryption. In this system the user, i.e., the student, generates a key pair on his own computer. This key pair can be used for authentication and can also be used in other situations. The authenticity of the key pair can be further validated by peer-to-peer authentication through the signing of other peoples' keys.

The students at the University of Copenhagen all have a unique username in the form of a ``KU Username''. If this username is mapped to a public key and the student holds the equivalent private key, then the authentication of an electronic hand-in has the same degree of trusted authenticity as that of the key pair.

We can use the KU login in order to establish this mapping and this mapping can be further authenticated by the above-mentioned peer-to-peer key signing although this latter authentication is not necessary for the system to work. In other words: support for peer-to-peer signing in this project is nice-to-have, not need-to-have. As a way to establish this mapping, the KU login is used as a trusted source of linking an identity to a public key. The role, if any, of peer-to-peer signing or key signing by teachers has not yet been established.


\subsubsection{Scenarios}\label{Scenarios}
\begin{itemize}
\item A student has just started on the course ``Maskinarkitektur'' (Processor Design). He learns that he has to hand in the assignment by signing it with an OpenPGP key. He must then generate his own key pair using a designated program like ``GPG4Win'' or through a webservice developed as a part of this project and hosted by KU. After the key pair has been generated, the student uploads his public key to a server and the server then stores the KU Username and the public key.
\item When the student hands in his assignment, he signs it with his newly generated private key. And the signature is then uploaded along with the assignment. A teacher is grading the assignments that have been handed in on the course ``Maskinarkitektur''. Before he grades them, he needs to check the validity of the identity of the student. This check can either be done automatically by software that interfaces with the system developed as part of this project or manually by the teacher.
\item A teacher signs feedback with an OpenPGP-key. The student then looks up the OpenPGP public key of the teacher and validates the signature.
\end{itemize}

\subsubsection{Solution Domain}\label{Solution Domain}

To solve the aforementioned scenarios, we have drafted a solution which employs a key-server that will act as an authority. It will contain OpenPGP-keys for all users of the system. To validate the users it will use their KU email, which is created by other parts of KU, and thereby validated. A mapping of these two components will be created as core database of the system.

As part of the system, a web-server will be established. It will facilitate the replacement of OpenPGP-keys in the case of loss of a users private key. This web-server also serves as point-of-contact for the initial creation of users. The aforementioned validation will be facilitated by an integrated email server.

Upon creation of a user, a unique link, containing all necessary info, will be generated and served to the users KU email. When this link is accessed by the user, they are offered to upload their own OpenPGP-keys, generated in dedicated software like ``PGP4Win'' or to create a new OpenPGP-key directly in the browser.

The replacement of the key will be processed in the same way, as the initial upload of a key, with the one exception, that the user manually has to request, that a new upload link will be sent to their KU email.

% :I

\subsubsection{Functional Requirements}\label{Functional_Requirements}
\begin{itemize}
\item The system must map a KU Username to a OpenPGP public key.
\item Users must be able to upload their own public key to the service.
\item If a user loses his private key, a method for replacing the public key must exist.
\item The initial registration of a key, and the replacement of a key, is authenticated through the KU email system.
\item Lookups in this table must be possible for the relevant university staff. An API will be defined and implemented for this.
\item The KU usernames must not be accessible for outsiders.
\item A guide for creating key pairs needs to be available on the website, possibly as shell script. This guide should explain how to derive an SSH key from an OpenPGP key.
\end{itemize}
\subsubsection{Nonfunctional Requirements}
\begin{itemize}
\item The backend should be written in golang.
\item The frontend should be Javascript and/or HTML5, which ever is safest and most easy to use.
\item The system should be browser independent.
\item The system must be compliant with all standards and regulations imposed by the Government of Denmark and the University of Copenhagen. These regulations will be within privacy and will regard the KU usernames that are part of the system.
\item The customer has decreed, that the software be licensed under a MIT-like license, which will be provided by the customer.
\end{itemize}
\subsubsection{Target Environment}
\begin{itemize}
\item The frontend must work on the newest editions of all mainstream browsers -- Internet Explorer, Firefox, Chrome, Safari.
\item The backend software must be able to run on the most popular Unix-like systems.
\end{itemize}
\subsubsection{Delivarables and Deadlines}\label{sec:deadlines}
\begin{itemize}
\item A preliminary version must be ready for testing and security auditing by mid May.
\item The whole project must be done by June 3$^{rd}$, 2015.
\item A list of documents to be delivered together with the software can be read in Section \ref{sec:deliverables}
\end{itemize}
\subsection{Initial Software Project Management Plan}
\subsubsection{Work Breakdown}
%lav det som træ!!one!!11!!
The following tasks have been identified. The list is subject to change.
\begin{itemize}
\item Acquire VPS
\item Initial set-up of VPS
\SubItem Install Go compiler
\SubItem Install SQLite or postgresql
\SubItem Create startup script for server to run appropriate programs.
\item Use golang to connect to database
\item Determine exact method for teachers to fetch data from DB API
\item Front-end view containing a form for first-time upload of OpenPGP key
\item Front-end view which enables replacement of key
\item Mail server to send links for students to upload OpenPGP key
\SubItem System that creates signed single-use links for key upload
\end{itemize}
\subsubsection{Roles \& Responsibilities}
%lav det som tabel (se bogen for stil)
%beskriv interne og eksterne roller
This group consists of Mads Gram, Sven Frenzel, and Thorkil Værge. The roles and responsibilities have not yet been delegated. So instead, we choose to describe our individual skill set.
\begin{itemize}
\item Mads Gram: Can code Javascript for front-end, some knowledge of MVC architecture, SQL and general DB knowledge
\item Sven Frenzel: Linux server administration, basic network administration. Has experience with HTML and CSS and is a proficient bash scripter.
\item Thorkil Værge: Some experience in project management of website development, SQL experience from DB course, some front-end experience with HTML and CSS. Theoretical knowledge of hashes and cryptography and practical knowledge of internet security.
\end{itemize}
\subsubsection{Project Schedule}
%lav gerne et gantt-diagram (ikke nødvendigt)
We have chosen to divide the project schedule into phases. In the spirit of agile development, the following list should be viewed as an iterative process and something that will overlap a lot. Some of these tasks could also be given a deadline under Section \ref{sec:deadlines}.
\begin{itemize}
\item Defining the project -- the introductory phase where we iteratively align our expectations for the project with the customer's expectations.
\item Learning Go -- this can be done through some very basic Go tutorials but will probably mainly be achieved by reading up on how a Go server is built and connects to a server as well as implementing this setup.
\item Obtaining additional required skill set: some front-end development in HTML5 or Javascript.
\item Setting up the server with required programs, including for example a Go compiler and an SQL server.
\item Programming the server and its SQL interaction in Go.
\item Programming the front-end.
\item Setting up email server to talk to Go on our VPS.
\item Presenting results to customer for feedback.
\item Testing the programmed logic.
\end{itemize}
\subsubsection{Budget and Resources}
The following resources are required for the project and will be provided by the customer.
\begin{itemize}
\item VPS or dedicated server.
\item access to a mail server (Oleks researches).
\item Subdomain on customer's server to host the PGP registration website.
\item SSL certificate to defeat man-in-the-middle attack where an attacker sends a fake public PGP key to the server.
\item time resources: As of yet unknown but we expect that large parts of the programming can be solved at intensive work weekends.
\end{itemize}
\subsubsection{Description of Risks}
%risikomatrice!
\begin{itemize}
\item There is a risk that learning Go will be too complex a task for us to master in time. To mitigate this risk, a deadline for some basic server functionality, using Go, should be made. We set a deadline for May 1$^{st}$ for getting the VPS to connect to the database through Go. This task requires basic Go knowledge and will test our basic Go abilities.
\item During the security audit, serious vulnerabilities could necessitate major restructuring of the code base.
\end{itemize}
\newpage
\subsection{Initial software architecture}
The purpose of the initial software architecture (ISA) is to provide an overview of the project by breaking down the system into smaller parts and explaining these individual parts. An overview of this is shown in Figure  \ref{fig:ISA}.

The core of the system is a database which is represented on the left of Figure  \ref{fig:ISA}. This database consists of a table which maps a KU email (or a KU Username) to a public PGP key. sqlite or postgresql will be used for manage the database.

The webserver interfaces with both the database and a mail server and also delivers the HTML to the user. The webserver is written in golang.

The user can land on two different pages depending on what they are after. One page is for uploading or generating a new key through Javascript and another is for replacing a key.

\begin{figure}[h!]
\centering
\includegraphics[width=1.1\textwidth]{pictures/pksu_isa_centered}
\caption{Initial software architecture of the project. Our project includes coding the database, web server, and front-end.}
\label{fig:ISA}
\end{figure}

\subsection{Project Agreement Definition}
The purpose of the Project Agreement Definition (PAD) is to outline the Project Agreement (PA) which is a contract that we, the developers, and the customer agree upon. A contract defining the duration, cost, deliverables, and minimum requirements for the project should be defined. The sections below are drafts of the contents of the PA which is yet to be fully defined and has \textbf{not} been approved of the customer. The next meeting should have the PA on the agenda.
\subsubsection{List of Deliverable Documents}\label{sec:deliverables}
\begin{itemize}
\item Source code of the project.
\item Technical documentation of the developed system.
\item Non-technical report describing the system in layman terms.
\item Scripts and/or guides for set-up of the developed server software.
\SubItem A report documenting set-up of all dependencies for the developed system, including, but not limited to, a log of all commands used to set-up the development environment.
\item Working server running compiled software.
\item Report of security audit.
\end{itemize}
\subsubsection{Criteria for Demonstrations of Functional Requirement}
All aspects of the basic functionality must be tested thoroughly. This includes tests of uploading of public keys, replacement of private keys, administrator access to the private keys and more.

\subsubsection{Test suite}

We will develop a test suite along with the system. This will be used internally to ensure that the logic is sound and that all functions run correctly. Furthermore it will also serve as a way to monitor the progress of the system.

Since multiple developers will be working on the code base, it is a genuine concern that changing the code can course problems throughout the code base. In that case, the test suite will check that such problems are not overlooked since the test suite will highlight the problems.

In the end, the test suite will be extended and be used as a validation of the system. The test suite will serve as the primary validation of the functional requirements.

In our specific case, some preliminary test will focus on correct execution of the individual cases. Then as the project evolves and becomes more interwoven, tests will be written also check the more complex functionality.

\subsubsection{Criteria for Demonstrations of Non-functional Requirements}
All aspects of the website should be tested and look uniform on all four standard web browsers.
Security-wise, the customer has expressed certain requirements. It must not be possible for non-approved persons to access the list of KU usernames since the university has not published this list and this then would consititute a loss of privacy for the students. It is however acceptable that the usernames are stored in plaintext on the server.
\section{System definition - FACTOR}
The system definition is described according to the FACTOR criteria described in \cite{matthiassen}. The following specifications are mostly redundant and reflects information placed elsewhere in the report. The purpose of this system definition is to give a quick overview of the requirements for the system to be designed.
\begin{itemize}
\item \textbf{F}unctionality: Allow specific students and employees at University of Copenhagen to map their username to a public PGP key and store these records in a database maintained by the customer. Also allow the same users to replace their public key in case they lose the corresponding private key. Let people with privileged access through some login mechanism read these mappings. Do not let the mappings be viewable to the general public.
\item \textbf{A}pplication domain: The system will be used by students to upload their PGP key and by teaching assistants to match a hand-in with a KU Username.
\item \textbf{C}onditions: Most students are not likely to have experience with PGP so it should be as simple as possible to upload a PGP key and the system must be forgiving, meaning that if a student loses his private PGP key, they must be able to upload a new one. The TA's must be able to retrieve the mappings in an easy way since they should not have to be bothered with SQL commands.
\item \textbf{T}echnology: Servers with a UNIX-like OS for back-end will run the system, modern browsers will be used for front-end. The system will be written in golang to the widest extent possible.
\item \textbf{O}bjects: OpenPGP public key, KU Username, Students, TAs, emails.
\item \textbf{R}esponsibility: Maintain a table with a mapping from KU Usernames to corresponding OpenPGP public keys. Allow privileged users access to read from this table.
\end{itemize}

\newpage
\section{Exercises from Weekly Assignments}

\subsection{OOSE 1-6:}
Specify which of these statements are functional requirements and which are
nonfunctional requirements
\begin{itemize}

\item “The TicketDistributor must enable a traveler to buy weekly passes.”\\
This first requirement is a \textbf{functional} requirement, since it is an essential function of the system.
\item “The TicketDistributor must be written in Java.”\\
This is a \textbf{non functional} requirement.\\\\
\item “The TicketDistributor must be easy to use.”\\
This is a \textbf{nonfunctional} requirement, an reason for this is that it can't be objectively answered. It is not possible too be simply evaluated as easy or hard to use.\\\\
\item “The TicketDistributor must always be available.”\\
This is a \textbf{functional} requirement. \\\\
\item “The TicketDistributor must provide a phone number to call when it fails.” \\
This is a \textbf{functional} requirement, since it describes a concrete function of the system.\\
\end{itemize}

\subsection{OOSE 1-8:}
In the following description, explain when the term account is used as an application domain concept and when as a solution domain concept.\\\\
In this specific example from the given text, we see the Application Domain is the issue of users not always being online. Which limits the customer to access to their account.\\
In this specific example from the given text, we see the Solution Domain, as the proposal of giving the user the data from the last connected session.\\
\newpage
\subsection{OOSE 2-6:}

\begin{figure}[h!]
    \centering
    \includegraphics[width=1.1\textwidth]{pictures/oose2_6}
    \label{fig:OOSE26}
\end{figure}

\subsection{OOSE 2-7:}
\begin{figure}[h!]
    \centering
    \includegraphics[width=1.1\textwidth]{pictures/oose2_7}

    \label{fig:OOSE27}
\end{figure}


\newpage
\subsection{OOSE 2-9:}

\begin{figure}[h!]
    \centering
    \includegraphics[width=1.1\textwidth]{pictures/oose2_9}

    \label{fig:OOSE29}
\end{figure}

\subsection{OOSE 2-10:}

\begin{figure}[h!]
    \centering
    \includegraphics[width=1.1\textwidth]{pictures/oose2_10}

    \label{fig:OOSE210}
\end{figure}

\newpage
\subsection{OOSE 7-1:}

\begin{figure}[h!]
    \centering
    \includegraphics[width=1.1\textwidth]{pictures/oose7_1}
    \label{fig:OOSE71}
\end{figure}

\newpage
\section{Appendix}
\subsection{Minutes from March 16th 2015}
\begin{itemize}
\item Ønsker: identificer studerende ved aflevering vha gpg-nøgle
\item Ønsker: grube med gpg-nøgle og ku-mail knyttet sammen (da der stoles på at den studerende har adgang til sin email)
\item Autentificering sker igennem unikke links til ku-mail (for at lægge gpg-nøgle op).
\SubItem 2 muligheder: copy/paste gpg-nøgle eller generer gpg-nøglepar direkte i browseren.
\item DIKU leverer ssl-certifikat.
\item Hvorfor ikke bruge eksisterende?
\item Public index j/n?
\item modtager ukrypteret nøgle og bruger den til at verificere indholdet - bagefter checkes hashet pubkey mod databasen.
\item Arbejdsbelastning: 10 timer per person per uger
\item database: (\underline{ku-nummerplade},hash(pubkey))
\end{itemize}
\subsubsection{Krav:}
\begin{itemize}
\item registrering gennem webformular og/eller kommandelinje
\item skal kunne genregistrering
\item anmodning kodkendes igennem ku-email
\item kunne vælge om den offentlige nøgle offentliggøres
\SubItem private by default
\item private API: få offentlig nøgle og returner brugernavn (kun adgang til DIKU-ansatte)
\item Sprog
\SubItem Backend: golang + sql
\SubItem frontend: html5 + js
\end{itemize}

næste møde:
2015-03-22 kl 11-16

\subsection{Minutes from March 22nd 2015}
The meeting was held to align the expectations of the client and the developers in regards to the systems capabilities and design.

\begin{itemize}
\item Oleks undersøger interface hvilke tilgangsmuligheder, der ønskes til databasen.
\item Kun lukket API, intet offentligt look-up. Der gemmes pupkey og abc123 (ku-brugernavn).
\item key-server i stil med MIT-keyserver er nice-to-have. Man bliver kun listet her, hvis man eksplicit ønsker dette.
\item Den studerende uploader en fil til afleveringssystemet (uden anden information). Afleveringssystemet anmoder om par af PubKey \& KU-ID og undersøger om filen er signeret af denne nøgle. Hvis ikke, så anmodes næste par indtil det korrekte par er fundet. Dette returneres til underviser.
\item Beta klar til start/midt maj.
\item Oleks gives a license (MIT-like)
\item ISA-diagram
\SubItem Fjern Admin-login fra (der er sshd-adgang til serveren anyways.)
\SubItem Front-end til client, er response og ikke request
\SubItem User guide /shell script ved ``Create key''
\item Oleks anmoder om at der sendes en dagsorden 24 timer før fremtidige møder
\item Thorkil foreslår at møder optages auditivt.
\end{itemize}

Næste møde er 29/30 marts 2015.

\begin{thebibliography}{1}

\bibitem{matthiassen} L. Mathiassen, A. Munk-Madsen, P.A. Nielsen, and J. Stage. {\em Object-Oriented Analysis \& Design} 2000: Marko Publishing House.

\bibitem{beck} Kent Beck et al. 2001 Agile Manifesto. [ONLINE] Available at: http://agilemanifesto.org/principles.html. [Accessed 19 April 15].

\bibitem{gould} Gould \& Lewis. (1985), Designing for usability: key principles and what designers think. Commun. ACM 28, 3 (March 1985), 300-311.

%\bibitem{impj}  The Japan Reader {\em Imperial Japan 1800-1945} 1973:
%Random House, N.Y.

%\bibitem{norman} E. H. Norman {\em Japan's emergence as a modern
%state} 1940: International Secretariat, Institute of Pacific
%Relations.

%\bibitem{fo} Bob Tadashi Wakabayashi {\em Anti-Foreignism and Western
%Learning in Early-Modern Japan} 1986: Harvard University Press.

\end{thebibliography}



\end{document}

